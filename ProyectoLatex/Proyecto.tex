\documentclass{article}

\usepackage{hyperref}
\usepackage{graphicx}

\begin{document}
	
	\begin{titlepage}
		\centering
		
		% Logo de la institución
		\includegraphics[width=1.2\textwidth]{logoTec.png}
		
		% Título del proyecto
		\vspace{2cm}
		{\LARGE Proyecto: Travelman Sales Problem \par}
		
		% Información del curso y profesor
		\vspace{1cm}
		{\large Curso: IC-3002. Análisis de Algoritmos \par}
		{\large Profesor: Eddy Ramírez Jiménez \par}
		
		% Nombre de los estudiantes
		\vspace{1cm}
		{\large Estudiantes: \par}
		\begin{tabular}{c}
			Samir Cabrera Tabash \\
			Mynell Myers Hall \\
		\end{tabular}
		
		% Fecha
		\vspace{1cm}
		{\large Fecha: \today \par}
	\end{titlepage}
	
	\tableofcontents
	\newpage
	
	\section{Resumen Ejecutivo}
	Este proyecto tiene como objetivo abordar el Problema del Viajante de Comercio (TSP) mediante la implementación y comparación de tres enfoques algorítmicos: backtracking, programación dinámica y algoritmos genéticos. La tarea se centra en encontrar el recorrido más corto que visite cada ciudad exactamente una vez y regrese al punto de inicio.
	
	
	faltan los resultados y las pruebas aca xd
	
	faltan los resultados y las pruebas aca xd
	
	faltan los resultados y las pruebas aca xd
	
	faltan los resultados y las pruebas aca xd
	
	faltan los resultados y las pruebas aca xd
	
	
	
	\section{Introducción}
	El presente proyecto aborda el fascinante y desafiante Problema del Viajante de Comercio (TSP), un clásico en la teoría de la computación y la optimización combinatoria. Este problema, formulado inicialmente en la década de 1800 y refinado a lo largo del tiempo, busca encontrar la ruta más corta que visite cada ciudad exactamente una vez y retorne al punto de inicio. Aunque su formulación matemática se estableció en la década de 1930, el TSP sigue siendo objeto de estudio en diversas disciplinas, incluyendo la informática, las matemáticas, la inteligencia artificial y la logística.
	
	El TSP ha demostrado ser un desafío computacional significativo, y su complejidad se acentuó cuando en 1972, Richard Karp demostró que es un problema NP-completo. Esta clasificación implica que no existe un algoritmo eficiente conocido para encontrar la solución óptima en todos los casos, a menos que se resuelva la famosa conjetura P = NP. Ante la dificultad de resolver el TSP de manera exacta para instancias grandes, han surgido numerosas heurísticas y algoritmos de aproximación que exploran enfoques como algoritmos genéticos, simulated annealing y algoritmos hormiga.
	
	El objetivo de este proyecto es explorar y comparar tres enfoques específicos para abordar el TSP: backtracking, programación dinámica y algoritmos genéticos. Cada uno de estos métodos presenta sus propias ventajas y limitaciones, y el análisis comparativo proporcionará una visión detallada de su rendimiento en términos de eficiencia y calidad de las soluciones.
	
	A lo largo de este documento, se detallarán las implementaciones de cada enfoque, destacando las estrategias algorítmicas utilizadas y los desafíos encontrados durante el desarrollo. 
	
	Este proyecto no solo busca proporcionar soluciones algorítmicas efectivas para el TSP, sino también fomentar la comprensión profunda de las técnicas empleadas y su aplicación en la resolución de problemas prácticos. La resolución del TSP no solo tiene implicaciones teóricas, sino también aplicaciones prácticas significativas en la planificación de rutas, la logística y la optimización de recursos.
	
	A través de este análisis detallado, esperamos no solo encontrar soluciones eficientes para el TSP, sino también contribuir al entendimiento y desarrollo continuo en el campo del análisis de algoritmos.
	
	\section{Marco Teórico}
	Java es un lenguaje de programación de propósito general que se ha vuelto extremadamente popular desde su introducción en la década de 1990. Aquí tienes una descripción de los antecedentes e historia del lenguaje de programación Java:
	
	\textbf{Antecedentes y Desarrollo Inicial:}
	\begin{itemize}
		\item \textbf{Orígenes en Sun Microsystems:} Java fue desarrollado en Sun Microsystems a principios de la década de 1990 por un equipo liderado por James Gosling, Mike Sheridan y Patrick Naughton. El proyecto inicialmente se llamó "Oak", pero más tarde se renombró a Java.
		
		\item \textbf{Objetivos de Diseño:} Los diseñadores de Java tenían la visión de crear un lenguaje de programación que fuera portátil, orientado a objetos, robusto y seguro. La portabilidad fue una de las características clave, permitiendo que el código escrito en Java pudiera ejecutarse en cualquier máquina virtual Java (JVM) sin importar la arquitectura subyacente.
	\end{itemize}
	
	\textbf{Hitos en la Historia de Java:}
	\begin{itemize}
		\item \textbf{1995 - Lanzamiento Oficial:} Java 1.0 fue lanzado al público en 1995. Este evento marcó el comienzo de la adopción generalizada de Java en la comunidad de desarrollo de software.
		
		\item \textbf{1996 - JVM en Especificación Estándar:} Sun Microsystems publicó las especificaciones para la Máquina Virtual Java (JVM), permitiendo a otros implementar sus propias versiones de la JVM.
		
		\item \textbf{1997 - JDK 1.1:} Java Development Kit (JDK) 1.1 fue lanzado, introduciendo mejoras significativas en el lenguaje y la plataforma, como las bibliotecas de eventos AWT y Swing.
		
		\item \textbf{2004 - Java 5 (J2SE 5.0):} Introdujo características importantes como anotaciones, genéricos, enumeraciones y el sistema de manejo de excepciones mejorado.
		
		\item \textbf{2011 - Adquisición por Oracle:} Oracle Corporation adquirió Sun Microsystems, y se convirtió en el mantenedor principal de Java.
		
		\item \textbf{2014 - Java 8:} Introdujo expresiones lambda, Stream API, y otras mejoras significativas en el lenguaje y la plataforma.
		
		\item \textbf{2017 - Java 9:} Introdujo el sistema de módulos, lo que facilita la modularidad en aplicaciones Java.
		
		\item \textbf{2018 - Java 11 (LTS):} Se estableció como una versión de soporte a largo plazo (LTS), marcando un cambio en el modelo de lanzamiento de Java con actualizaciones regulares y versiones LTS.
		
		\item \textbf{2020 - Java 14 y 15:} Introdujo mejoras en el manejo de patrones, registros y otras características.
	\end{itemize}
	
	\textbf{Características y Fortalezas de Java:}
	\begin{itemize}
		\item \textbf{Portabilidad:} El lema "Write Once, Run Anywhere" refleja la capacidad de ejecutar el código Java en diversas plataformas sin necesidad de recompilación.
		
		\item \textbf{Orientación a Objetos:} Java sigue el paradigma de programación orientada a objetos, facilitando la creación de software modular y reutilizable.
		
		\item \textbf{Seguridad y Robustez:} Java fue diseñado con un énfasis en la seguridad y la prevención de errores en tiempo de ejecución.
		
		\item \textbf{Gran Ecosistema:} Java cuenta con una amplia variedad de bibliotecas y frameworks que facilitan el desarrollo de aplicaciones en diferentes dominios.
		
		\item \textbf{Comunidad Activa:} Java tiene una comunidad de desarrollo activa y una amplia base de usuarios.
	\end{itemize}
	
	La historia de Java está marcada por su evolución continua para adaptarse a las cambiantes necesidades del desarrollo de software. Aunque ha habido desarrollos y desafíos a lo largo de los años, Java sigue siendo una opción popular para una variedad de aplicaciones, desde desarrollo web hasta sistemas embebidos y móviles.
	
	En el desarrollo de este proyecto, se han empleado varias clases y interfaces proporcionadas por la biblioteca estándar de Java, especialmente dentro del paquete \texttt{java.util}. A continuación, se presenta una breve descripción de algunas de las clases utilizadas:
	
	\begin{itemize}
		\item \textbf{\texttt{Scanner:}} La clase \texttt{Scanner} se ha empleado para facilitar la lectura de datos de entrada desde la consola. Proporciona métodos para analizar tipos de datos primitivos y cadenas de texto.
		
		\item \textbf{\texttt{ArrayList:}} La clase \texttt{ArrayList} se ha utilizado para la implementación de estructuras de datos dinámicas basadas en listas. Permite el almacenamiento y manipulación eficiente de elementos de manera dinámica.
		
		\item \textbf{\texttt{List:}} La interfaz \texttt{List} se ha empleado como base para la manipulación de listas ordenadas. Proporciona métodos comunes para la manipulación de elementos en una secuencia.
		
		\item \textbf{\texttt{HashMap:}} La clase \texttt{HashMap} se ha utilizado para implementar mapas clave-valor. Es útil para asociar valores con claves y permite un acceso rápido a los datos.
	\end{itemize}
	
	Estas clases y interfaces de la biblioteca \texttt{java.util} han sido fundamentales para la implementación exitosa de diversas funcionalidades en este proyecto.
	
	\section{Descripción de la Solución}
	% Aquí va la descripción de la solución
	
	\section{Resultados de Pruebas}
	% Aquí van los resultados de las pruebas
	
	\section{Conclusiones}
	% Aquí van las conclusiones
	
	\section{Aprendizajes}
	
	La realización de este proyecto nos brindó una valiosa experiencia en la implementación y análisis de algoritmos para abordar el Problema del Viajante de Comercio (TSP). Utilizando Java como lenguaje de programación, hemos extraído aprendizajes significativos tanto a nivel individual como colectivo.
	
	\subsection{Dominio de Java}
	
	Programar en Java nos permitió fortalecer nuestras habilidades en este lenguaje orientado a objetos. Desde la gestión de clases y objetos hasta la manipulación de estructuras de datos, el proyecto consolidó nuestro dominio de Java y su aplicación en la resolución de problemas algorítmicos complejos.
	
	\subsection{Implementación de Algoritmos}
	
	La implementación de distintos enfoques para el TSP, como backtracking, programación dinámica y algoritmos genéticos, nos proporcionó una comprensión profunda de las estrategias algorítmicas específicas para cada caso. La experiencia práctica en la traducción de conceptos teóricos a código funcional enriqueció nuestra perspectiva sobre la eficiencia y la complejidad algorítmica.
	
	\subsection{Paralelización y Eficiencia}
	
	La integración de la paralelización en Java para ejecutar los algoritmos de manera concurrente nos introdujo en el mundo de la optimización de procesos. Aprendimos a gestionar la concurrencia de manera efectiva, mejorando la eficiencia de nuestros programas y comprendiendo la importancia de la adaptabilidad en entornos de ejecución paralela.
	
	\subsection{Benchmarking y Evaluación de Resultados}
	
	La realización de pruebas exhaustivas y la evaluación comparativa de los resultados nos enseñaron la importancia del benchmarking en la validación de algoritmos. Aprendimos a interpretar y analizar datos de rendimiento, lo cual es esencial para la toma de decisiones informadas en el diseño y selección de algoritmos.
	
	\subsection{Colaboración en Equipo}
	
	Trabajar en parejas nos brindó la oportunidad de compartir conocimientos, abordar desafíos de manera colaborativa y aprovechar la diversidad de habilidades individuales. La comunicación efectiva fue clave para el éxito del proyecto, destacando la importancia de la colaboración en equipo.
	
	\subsection{Aplicaciones Prácticas y Relevancia Teórica}
	
	Comprender la aplicabilidad del TSP en contextos prácticos, como la planificación de rutas y la logística, nos conectó con las aplicaciones del análisis de algoritmos en situaciones del mundo real. Al mismo tiempo, la profundización en la teoría del TSP contribuyó a nuestro entendimiento de problemas computacionales complejos.
	
	En resumen, la realización exitosa de este proyecto en Java no solo amplió nuestras habilidades técnicas, sino que también enriqueció nuestra comprensión de los principios fundamentales de la ciencia de la computación y la teoría de algoritmos. Estos aprendizajes se traducen en una base sólida para abordar futuros desafíos en el ámbito de la programación y el análisis de algoritmos.
	
	
	\section{Bibliografía}
	\begin{thebibliography}{9}
		\bibitem{fcc_graph_theory}
		freeCodeCamp.org,
		\emph{Algorithms Course - Graph Theory Tutorial from a Google Engineer},
		9 de octubre de 2019.
		\url{https://www.youtube.com/watch?v=09_LlHjoEiY}.
		Accedido el 3 de enero de 2024.
		
		\bibitem{reducible_tsp}
		Reducible,
		\emph{The Traveling Salesman Problem: When Good Enough Beats Perfect},
		26 de julio de 2022.
		\url{https://www.youtube.com/watch?v=GiDsjIBOVoA}.
		Accedido el 3 de enero de 2024.
		
		\bibitem{emami_taba_tsp}
		M. Emami Taba,
		\emph{Solving Traveling Salesman Problem With a Non-complete Graph},
		Tesis de Maestría,
		Univ. Waterloo,
		Waterloo, Ontario, Canada,
		2009.
	\end{thebibliography}
	
\end{document}
